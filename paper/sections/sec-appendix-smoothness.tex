% !TEX encoding = ISO-8859-16
\section{Proof of Theorem~\ref{thm-sliced-energy-grad}}
\label{sec-proof-thm-sliced}


\noindent\textbf{Notations.}  
Without loss of generality, for a fixed $Y \in \RR^{d \times N}$, we study the smoothness of 
\eq{
	\foralls X \in \RR^{d\times N}, \quad
	\Ee(X) = \frac{1}{2} \SWass{\RR^d}(\mu_X,\mu_Y)^2 = \int_{\Sph} \Ee_\th(X) \d x
}
\eq{
	\qwhereq
	\Ee_\th(X) = \frac{1}{2} \WassPt(X_\th,Y_\th)^2.
}
We have used, for $x, y \in \RR^N$, the shorthand notation
\eq{
	\WassPt(x,y) = \Wass{\RR}(\mu_x,\mu_y).
}
The result of Theorem~\ref{thm-sliced-energy-grad} then follows by summations of such functionals. 

We define $\UU(N,d)$ to be vectors of $\RR^{d \times N}$ with distinct entries:
\eq{
	\UU(N,d) = \enscond{
				 W = (W_1, \ldots, W_N) \in \R^{d \times N} 
			}{
					\foralls i \not=j,\, X_i \not= X_j
	 		 }.
}
The hypothesis is that $X \in \UU(N,d)$. 
One has 
\eq{
	\Ee_\th(X) = \frac{1}{2} \norm{X_\th - Y_\th \circ \si_\th }^2
	\qwhereq
	\si_\th = \si_X^\th \circ (\si_Y^{\th})^{-1}
}
is a permutation depending on both $X$ and $Y$. Note that the permutation involved are not necessarily unique, and are assumed to be arbitrary valid sorting permutations.

For $X \in \RR^{N \times d}$ and $\epsilon>0$ we introduce 
\eq{
	\Theta_\epsilon(X) = \enscond{ \th \in \Sph}{ 
			\foralls \normP{\de} \leq \epsilon, \quad
			X_\th + \de_\th \in \UU(N,1)
	}.
}
This is the set of directions for which any perturbation of $X$ of amplitude smaller than $\epsilon$ has a projection with disjoint points. 

\medskip
\noindent\textbf{Overview of the proof.}  
In the following, we thus aim at proving that $\Ee$ is $C^1$, that
\eq{
	\tilde\nabla \Ee(X) = \int_{\Sph} \tilde \nabla \Ee_\th(X) \d\th
	\qwhereq
	\tilde \nabla \Ee_\th(X) = (X_\th - Y_\th \circ \si_\th) \th
}
is indeed equal to $\nabla \Ee(X)$, and that this gradient is Lipschitz continuous. 

The general strategy of the proof is to split the integration between the directions $\th \in \Theta_\epsilon(X)$, for which we can locally assume that the permutations $\si_\th$ are constant (see Lemma~\ref{lem-permutation-unique}), which in turn defines a smooth quadratic energy, and the remaining directions in $\Theta_\epsilon(X)^c$, which are shown to have a negligible contribution to the energy and to the derivative (see Lemma~\ref{lem-volume}). 

\medskip
\noindent\textbf{Preparatory results.}  The following lemma shows that if $\th \in \Theta_\epsilon(X)$ the permutations $\si_X^\th$ are stable to small perturbations of $X$.

\begin{lem}\label{lem-permutation-unique}
	Let $X \in \UU(N,d)$.
	For all $\th \in \Theta_\epsilon(X)$, for all $\de$ with $\normP{\de} \leq \epsilon$, the permutation 
	$\si_{X+\de}^\th$ that sorts $( \dotp{X_i+\de_i}{\th} )_i$
	is uniquely defined and
	satisfies $\si_{X+\de}^\th = \si_X^\th$.
\end{lem}
\begin{proof}
	If one has  $\si_{X+\de}^\th \neq \si_X^\th$, then necessarily there exists some $t \in [0,1]$ such that 
	$\si_{X+t\de}^\th$ is not uniquely defined, which is equivalent to 
	$X_\th+t\de_\th$ not being in $\UU(N,1)$. Since $\normP{t \de} \leq \epsilon$, this shows
	that $\th \notin \Theta_\epsilon(X)$.
\end{proof}

In order to prove Theorem \ref{thm-sliced-energy-grad}, we need the following lemma.

\begin{lem}\label{lem-volume}
	For $X \in \UU(N, d)$, one has
	\eql{\label{eq-control-theta}
		\text{\upshape Vol}(\Theta_\epsilon(X)^c) = \int_{\Theta_\epsilon(X)^c} \d \th 
		= O( \epsilon ).
	}
\end{lem}

\begin{proof}
	One has $X_\th + \de_\th \notin \UU(N,1)$ if and only there exists a pair of points
	$u=X_i+\de_i$ and $v=X_j+\de_j$ with $i \neq j$ such that
	\eq{
		\th \in A(u,v) 
		\qwhereq
		A(u,v) = \enscond{\xi \in \Sph}{\dotp{\xi}{u-v}=0}
	}	
	Note that $A(u,v)$ is a great circle of the sphere $\SS^{d-1}$. 
	 
	One can thus covers $\Theta_\epsilon(X)^c$ using the union of all such circles $A(u,v)$, which shows
	\eq{
		\Theta_\epsilon(X)^c \subset \bigcup_{i \neq j} A_\epsilon(X_i,X_j) \qwhereq
		A_\epsilon(x,y) = 
		\bigcup_{ {\scriptsize \begin{matrix} \norm{u-x} \leq \epsilon	
			\\ \norm{v-y} \leq \epsilon \end{matrix}} }
			A(u,v)
	}
	Note that the geodesic distance $d$ on the sphere $\SS^{d-1}$ between two circles is equal to the angle between the normal to the planes of the circles
	\eq{
		d(A(u,v),A(x,y)) = \text{Angle}(u-v,x-y)
		 = \text{Angle}(x-y + \epsilon w,x-y)
	}
	where $\norm{w}\leq 2$.
	As $\epsilon \rightarrow 0$, after some computations, 
	one has the following asymptotic decay of the angle
	\eq{
		\text{Angle}(x-y + \epsilon w,x-y) = O(\epsilon/\norm{x-y})
	}
	and thus 
	$d(A(u,v),A(x,y)) \leq C \epsilon$ for some constant $C$. This proves that 
	$\foralls u,v$, one has
	\eq{
		\choice{
			\norm{u-x} \leq \epsilon \\
			\norm{v-y} \leq \epsilon 
		}
		\qarrq	
		A(u,v) \subset B_{C\epsilon}(x,y)
	}
	for some constant $C>0$, where
	\eq{
		B_{\epsilon}(x,y) = \enscond{ \xi \in \Sph }{ d(\xi,A(x,y)) \leq \epsilon }
	}
	One thus has 
	\eq{
		A_{\epsilon}(x,y) \subset B_{C\epsilon}(x,y).
	}  
	The volume of the spherical band $B_{C\epsilon}(x,y)$ of width $C\epsilon$ is proportional to $\epsilon$,
	and thus Vol$(A_\epsilon(x,y)) = O(\epsilon)$.
	Since $\Theta_\epsilon(X)^c$ is a finite union of such sets, one obtains the result.
\end{proof}

\noindent\textbf{Proof of continuity.} For each $\th$, the function $\Ee_\th$ is continuous as a minimum of continuous functions. The function $\Ee$ being an integral of $\Ee_\th$ on a compact set $\Sph$, it is thus continuous.

\medskip
	
\noindent\textbf{Proof of differentiability.}  Let $\de \in \RR^{N \times d}$ and $\epsilon = \normP{\de}$. The definition of the Wasserstein distance reads
	\eq{
		\WassPt((X+\de)_\th,Y_\th)^2 = 
		\norm{ (X_\th + \de_\th) \circ \si_{X+\de}^\th - Y_\th \circ \si_Y^\th }^2.
	}	
	For all $\th \in \Theta_\epsilon(X)$, thanks to Lemma~\ref{lem-permutation-unique}, 
	$\si_{X+\de}^\th = \si_{X}^\th$.
	One can thus compute the variation of the 1-D Wasserstein distance with respect to $\de$ as			
	\begin{align}
		\WassPt((X+\de)_\th,Y_\th)^2 &=
		\norm{ X_\th+\de_\th - Y_\th \circ \si_\th }^2 \\
		\label{eq-proof-appendix-1} &= \WassPt(X_\th,Y_\th)^2 + \dotpP{ \tilde\nabla \Ee_\th(X) }{ \de } + \norm{\de_\th}^2.
	\end{align}
	Note that the fact that $\si_Y^{\th}$ might not be uniquely defined has no impact on the value of~\eqref{eq-proof-appendix-1}.
	One thus has 
	\eq{
		\Ee(X+\de)-\Ee(X) - \dotpP{\tilde\nabla \Ee(X)}{\de} = A(\de) + B(\de) + O(\normP{\de}^2)
	}
	where
	\begin{align*}
		A(\de) &= \int_{\Theta_\epsilon(X)^c} \pa{			
			\WassPt(X_\th+\de_\th,Y_\th)^2 - \WassPt(X_\th,Y_\th)^2
		} \d \th \\
		\qandq
		B(\de) &= -\int_{\Theta_\epsilon(X)^c} \dotpP{ \tilde\nabla \Ee_\th(X) }{ \de } \d \th 		
	\end{align*}
	Note that in the expression of $B(\de)$ the permutation $\si_\th$ involved in $\tilde\nabla \Ee_\th(X)$ is not necessary unique, and can be chosen arbitrarily. 
	
	One has, 
	\eq{
		|\dotpP{ \tilde\nabla \Ee_\th(X) }{ \de }| \leq
		\normP{X-Y\circ\sigma^\th} \normP{\de}
	}
	which implies, using Lemma \ref{lem-volume}
	\eql{\label{eq-proof-thm-1}
		|B(\de)| \leq O( \text{Vol}(\Theta_\epsilon(X)^c) \normP{\de} )
		= O(\normP{\de}^2) = o(\normP{\de}).
	}
	
	Since $(\th,X) \mapsto \Ee_\th(X)$ is  continuous and defined on a compact set, 
	it is uniformly continuous, and thus 
	\eq{
		|\WassPt(X_\th+\de_\th,Y_\th)^2 - \WassPt(X_\th,Y_\th)^2|
		\leq C(\de)
	}
	where $C(\de) \rightarrow 0$ where $\de \rightarrow 0$. This shows that
	\eql{\label{eq-proof-thm-2}
		|A(\de)| \leq \text{Vol}(\Theta_\epsilon(X)^c) C(\de) = o(\normP{\de}).
	}
	
	Putting together \eqref{eq-proof-thm-1} and \eqref{eq-proof-thm-2} leads to 
	\eq{
		|\Ee(X+\de)-\Ee(X) - \dotp{\tilde\nabla \Ee(X)}{\de}| =
		 o(\normP{\de})		
	}	
	which shows that $\Ee$ is differentiable with $\nabla \Ee = \tilde\nabla \Ee$.
	
\medskip
\noindent\textbf{Proof of Lipschitzianity of the gradient.}	
For all $\th \in \Th_0(X)$, $\nabla \Ee_\th(X)$ is continuous and uniformly bounded, and thus $\nabla \Ee$ is continuous. One has, for $\de \in \RR^{N \times d}$, and denoting $\epsilon=\norm{\de}$, 
	\eq{
		\nabla \Ee(X+\de) - \nabla \Ee(X) = M( \Th_\epsilon(X) ) + M( \Th_\epsilon(X)^c ) 
	}
	\eq{
		\qwhereq
		M(U) = \int_U ( \nabla \Ee_\th(X+\de) - \nabla \Ee_\th(X) ) \d \th.	
	}
	One has 
	\eq{
		M( \Th_\epsilon(X) ) = \int_{ \Th_\epsilon(X) } \de_\th \th  \d \th
	}
	whereas
	\eq{
		M( \Th_\epsilon(X)^c ) = \int_{ \Th_\epsilon(X)^c } \de_\th \th  \d \th
		+ \int_{ \Th_\epsilon(X)^c } ( Y \circ \tilde \si_\th - Y \circ \si_\th ) \th \d \th
	}
	where $\tilde \si_\th = \si_{Y_\th} \circ \si_{X_\th + \de_\th}^{-1}$.
	Using Lemma~\eqref{lem-volume}, one has for some constant $C>0$,  $\text{\upshape Vol}(\Theta_\epsilon(X)^c) \leq C \normP{\de}$ and hence
	\eq{
		\normP{ \nabla \Ee(X+\de) - \nabla \Ee(X) }
		\leq 
		(1 + 2 C \normP{Y})\normP{\de} 
	}
	which shows that $\nabla \Ee$ is $(1 + 2 C \normP{Y})$-Lipschitz continuous. 
		


