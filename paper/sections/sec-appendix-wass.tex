\section{Proofs of Section~\ref{sec-bary-wass}}
\label{sec-appendix-wass}

\begin{proof}[Proof of Proposition~\ref{prop-invariance-W}]
	From the definition~\eqref{eq-dfn-wass-dist}, one verifies that 
	\eql{\label{eq-inv-wass-dist}
		\Wass{\RR^d}(\phi_{s,u} \sharp \mu_1,\phi_{s,u} \sharp \mu_2)
		= 
		s \Wass{\RR^d}(\mu_1,\mu_2).
	}	
	so that
	\begin{align*}
		\Ee_{s,u}(\mu) &= \sum_{i \in I} \la_i \Wass{\RR^d}( \phi_{s,u} \sharp \mu_i, \mu )^2\\
		&= s^2 \sum_{i \in I} \la_i \Wass{\RR^d}( \mu_i, \phi_{s,u}^{-1} \sharp \mu )^2
		= s^2 \Ee_{1,0}(\tilde \mu).
	\end{align*}
	where we have introduced the following change of variable 
	\eq{
		\mu = \phi_{s,u} \sharp \tilde\mu
		\quad\Longleftrightarrow\quad
		\tilde \mu = \phi_{s,u}^{-1} \sharp \mu, 
	}
	(note that $\phi_{s,u}^{-1} = \phi_{s^{-1},-s^{-1}u}$).
	One thus has
	\begin{align*}
		\uargmin{ \mu } \Ee_{s,u}(\mu) &=
		\phi_{s,u} \sharp  \uargmin{ \tilde\mu } \Ee_{1,0}(\tilde\mu) 
	\end{align*}
	which proves~\eqref{eq-prop-inv-1}.
	Property~\eqref{eq-prop-inv-rot} is proved similarly. Properties~\eqref{eq-prop-inv-rad} and~\eqref{eq-prop-inv-cent} directly follow from ~\eqref{eq-prop-inv-rot}.
\end{proof}


\begin{proof}[Proof of Proposition~\ref{prop-invariance-W-bis}]
	We aim at determining $(s^\star,u^\star)$ such that 
	\eq{
		\mu^\star \in \Bary{\RR^d}^W(\mu_i,\la_i)_{i \in I} 
		\qwhereq 
		\choice{
			\mu^\star = \phi^\star \sharp \mu, \\
			\mu_i = \phi_i \sharp \mu, 
		}
	}
	and where for simplicity we have denoted $\phi_i = \phi_{s_i,u_i}$ and $\phi^\star = \phi_{s^\star,u^\star}$.	  	
	First, let us notice that 
	\eq{
		\phi_{s,u}(x) = \nabla \pa{ \frac{s}{2}\norm{x+u/s}^2 },
	} 
	so that the set $\Tt$ of maps of the form $\phi_{s,u}$ is a subset of gradients of convex functions.
	This point is important since optimal maps between $\mu_i$ and $\mu^\star$ are characterized as the gradient of convex functions that push forward $\mu_i$ onto $\mu^\star$, see~\cite{Villani03}. 
	Following~\cite{Carlier_wasserstein_barycenter}, we thus only need to show that
	\eq{
		\sum_{i \in I} \la_i T_i = \Id_{\RR^d} 
		\qwhereq
		T_i = \phi^\star \circ \phi_i^{-1}
		= \phi_{ \tilde s_i, \tilde u_i }
	}
	\eq{
		\qwhereq
		\choice{
			\tilde s_i = s^\star s_i^{-1} \\
			\tilde u_i = u^{\star}-s^\star s_i^{-1} u_i
		}
	} 
	since $T_i \sharp \mu_i = \mu^\star$ and $T_i \in \Tt$ is a gradient of a convex function. 
	So that $\mu^\star$ is a barycenter if and only if
	\begin{align*}
		\sum_{i \in I} \la_i T_i & = \sum_{i \in I} \la_i \phi_{\tilde s_i, \tilde u_i} \\
		& =  \phi_{\sum_{i \in I} \la_i \tilde s_i, \sum_{i \in I} \la_i \tilde u_i } 
		= \Id_{\RR^d} = \phi_{1,0}.
	\end{align*}
	This in turn is equivalent to the relationships 
	\eq{
		\sum_{i \in I} \la_i \tilde s_i = 1
		\qandq 
		\sum_{i \in I} \la_i \tilde u_i  = 0, 
	}
	which corresponds to~\eqref{prop-invariance-W-bis-formula}. 
\end{proof}

\begin{proof}[Proof of Proposition~\ref{prop-bary-1d-pushfwd}]
	The proof is done in~\cite{Carlier_wasserstein_barycenter} for $\mu = \mu_j$ for some $j \in I$, which is supposed to be absolutely continuous. It extends to an arbitrary measure $\mu$.
\end{proof}


\begin{proof}[Proof of Corollary~\ref{prop-bary-1d}]
	When using $\mu$, the uniform and normalized measure on $[0,1]$, with the notation of Proposition~\eqref{prop-bary-1d-pushfwd}, one has $T_i = C_{\mu_i}^+$. This is indeed a classical result for 1-D optimal transport, see for instance~\cite{Carlier_wasserstein_barycenter}, Section~6.1. 	
	One then recognizes that formula~\eqref{eq-bary-1d-formula-deriv} is the same as formula~\eqref{eq-bary-1d-pushfwd}.
\end{proof}

\begin{proof}[Proof of Proposition~\ref{prop-bary-omegad-variational}] One has
\begin{align*}
	\nu^\star & \in \uargmin{\nu \in \bar\Mm_1^+(\Om^d)} \sum_{i \in I} \la_i \Wass{\Om^d}(\nu_i,\nu)^2  	\\
	 & = \uargmin{\nu \in \bar\Mm_1^+(\Om^d)} \int_{\Sph} \sum_{i \in I} \la_i \Wass{\RR}( \nu_i^\th, \nu^\th )^2 \d \th.
\end{align*}
This is equivalent to the fact that for almost all $\th \in \Sph$, one has
\eq{
	\nu^{\star, \theta} \in \Bary{\RR}^W(\nu_i^\th,\la_i)_{i \in I}.
}
\end{proof}


\begin{proof}[Proof of Proposition~\ref{prop-bary-omd}]
	%%%
	\textit{Proof of~\eqref{propBar1}.} Similarly to the proof of~\eqref{eq-prop-inv-1}, the proof of~\eqref{propBar1} is obtained by using the following invariance of the Wasserstein distance on $\Om^d$
	\begin{align}\label{eq-inv-wass-omd}
		\Wass{\Om^d}(  \psi_{s,u} \sharp \nu_1, \psi_{s,u} \sharp \nu_2 ) = 
		s  \Wass{\Om^d}( \nu_1, \nu_2 ).
	\end{align}	
	%%%	
	\textit{Proof of~\eqref{propBar2}.} One has that 
	$\nu^\star \in \Bary{\Om^d}^W( \psi_{s_i,u_i} \sharp \nu, \la_i )_{i \in I}$ is equivalent to
	\eq{
		\text{for almost all $\th \in \Sph$},
		%\foralls \th \in \Sph, 
		\quad
		(\nu^\star)^\th \in 
		\Bary{\RR}^W( \phi_{s_i,\dotp{u_i}{\th}} \sharp \nu^\th, \la_i )_{i \in I}.
	} 
	Using the property of proposition~\ref{prop-invariance-W-bis} for $d=1$, one obtains that 
	\eq{
		\Bary{\RR}^W( \phi_{s_i,\dotp{u_i}{\th}} \sharp \nu^\th, \la_i )_{i \in I}
		% = Julien : j'ai enlev� l'�galit�
		\;\ni\;
		\phi_{s^\star,\dotp{u^\star}{\th}} \sharp \nu^\th, 
	} 
	which gives the desired result. 	
\end{proof}
